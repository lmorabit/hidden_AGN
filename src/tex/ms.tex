% Define document class
\documentclass[usenatbib,fleqn]{mnras}
\usepackage{showyourwork}
\usepackage{graphicx}
\usepackage{amssymb,amsmath}
\usepackage{color}
\usepackage{multirow}
\usepackage{soul}
\usepackage{orcidlink}

\newcommand{\sarc}{$^{\prime\prime}\!\!$}
\newcommand{\myreferences}{references.bib}
\newcommand{\frii}{FR$\,$II}
\newcommand{\fri}{FR$\,$I}
\newcommand{\wphz}{$\,$W$\,$Hz$^{-1}$}
\newcommand{\tbpeak}{$T_{b,\mathrm{peak}}$}
\newcommand{\tbtotal}{$T_{b,\mathrm{total}}$}

% Title
\title[TBD]{TBD}

% Author list
\author[L.K. Morabito]{\parbox{\textwidth}{Leah K. Morabito$^{1,2}$\thanks{E-mail: leah.k.morabito@durham.ac.uk}\orcidlink{0000-0003-0487-6651}\\}\\ 
$^{1}$Centre for Extragalactic Astronomy, Department of Physics, Durham University, Durham DH1 3LE, UK \\
$^{2}$Institute for Computational Cosmology, Department of Physics, Durham University, South Road, Durham DH1 3LE, UK \\ }

% Begin!
\begin{document}

\date{}
\pagerange{\pageref{firstpage}--\pageref{lastpage}} \pubyear{2021}
\maketitle

\label{firstpage}


% Abstract with filler text
\begin{abstract}
    TBD
\end{abstract}

% Main body with filler text
\section{Introduction}
\label{sec:intro}
The extent to which active galactic nuclei (AGN) impact galaxy formation is a major open question in astrophysics. It is clear that there is an interaction, from both observational and theoretical standpoints. Observations have revealed tight scaling relations between the mass of a super-massive black hole and its host galaxy \citep[see, e.g.][and references therein]{kormendy_coevolution_2013}, while cosmological simulations require some form of AGN feedback \citep{bower_breaking_2006,croton_many_2006} to be able to reproduce the observed galaxy population. Galaxies build up their stellar mass through star formation, while super-massive black holes are expected to provide either negative or positive feedback \hl{citations!} to either suppress or stimulate galaxy growth. 

To understand the interplay between AGN activity and star formation (SF), it is necessary to measure both processes. 

Stuff and things \citep{morabito_sub-arcsecond_2022}.

\section{Data and Methods}



\section{Results}

\begin{figure}
    \centering
    \includegraphics[width=0.5\textwidth]{figures/simba_comparison.png}
    \caption{Caption}
    \label{fig:simba}
    \script{simba_comparison.py}
\end{figure}

\begin{figure*}
    \centering
    \includegraphics[width=0.98\textwidth]{figures/deep_fields_RLFs.png}
    \caption{Caption}
    \label{fig:msrlfs}
    \script{comparison_plot.py}
\end{figure*}

\begin{figure*}
    \centering
    \includegraphics[width=0.98\textwidth]{figures/RLF_evolution_AGN.png}
    \caption{Caption}
    \label{fig:agn_evolution}
    \script{RLF_evolution.py}
\end{figure*}

\begin{figure*}
    \centering
    \includegraphics[width=0.98\textwidth]{figures/RLF_evolution_SF.png}
    \caption{Caption}
    \label{fig:sf_evolution}
    \script{RLF_evolution.py}
\end{figure*}


\section{Discussion and conclusions}


\section*{Data Availability}
Reference showyourwork on github

\section*{Acknowledgements}
LKM is grateful for support from UKRI [MR/T042842/1]. \hl{also cosma and spider?}

\bibliographystyle{mnras}
\bibliography{references}

\label{lastpage}

\end{document}
